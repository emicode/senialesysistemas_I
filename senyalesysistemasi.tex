\documentclass{myarticle}
% Preambulo
\usepackage[utf8]{inputenc}
\usepackage[spanish]{babel}
\def\lineach#1#2{
#1\hfill #2
}
\begin{document}
%\begin{center}{\huge{\bf SE\~{N}ALES Y SISTEMAS I.}}\end{center}
\begin{center}{\huge{\bf SEÑALES Y SISTEMAS I.}}\end{center}

\begin{description}
\item[{\large{\bf I.}}]{\large{\bf PROPÓSITO.}}\\
%\noindent 
PROPORCIONAR AL PERSONAL DISCENTE LAS TÉCNICAS DE ANÁLISIS DE SEÑALES Y 
SISTEMAS DISCRETOS, TANTO EN EL DOMINIO DEL TIEMPO COMO DE LA FRECUENCIA 
QUE LE PERMITAN DISEÑAR SISTEMAS DE COMUNICACIÓN DE DATOS,  \ EVI\-DEN\-CIAN\-DO 
LOS VALORES FUNDAMENTALES DEL EJÉRCITO Y FUERZA AÉREA MEXICANO.
\item[{\large{\bf II.}}]{\large{\bf ALCANCES.}}\\
%\noindent 
LA ASIGNATURA DE SEÑALES Y SISTEMAS I ES UNA ASIGNATURA ANTECEDENTE DE 
LA ASIGNATURA DE SEÑALES Y SIS\-TE\-MAS II, SIENDO UNA ASIGNATURA DE LAS 
CIENCIAS DE LA INGENIERÍA; EN LOS TEMAS CONTEMPLADOS EN LA ASIGNATURA, 
SE EMPLEAN ASPECTOS DESDE CONCEPTOS DE CONVERSIÓN A/D INCLUYENDO MUESTREO, 
CUANTIFICACION, CODIFICACIÓN Y CÁLCULOS DE RELACIÓN SEÑAL A RUIDO. 
TÉCNICAS DE ANÁLISIS DE SEÑALES Y DE SISTEMAS DISCRETOS EN EL DOMINIO DEL 
TIEMPO. TÉCNICAS DE ANÁLISIS DE SEÑALES Y DE SISTEMAS DISCRETOS EN EL 
DOMINIO DE LA FRECUENCIA. TRANSFORMADA Z Y SU APLICACIÓN AL ANÁLISIS DE 
SISTEMAS DISCRETOS. HASTA UNA INTRODUCCIÓN AL DISEÑO DE FILTROS DIGITALES.
\item[{\large{\bf III.}}]{\large{\bf  METODOLOGÍA DEL TRABAJO.}}\\
\begin{description}
\item[A.]EL PERSONAL DOCENTE UTILIZARÁ EL MÉTODO INDUCTIVO DURANTE LA 
EXPOSICIÓN DE LOS TEMAS Y RESOLUCIÓN DE PROBLEMAS SENCILLOS, ASIGNARÁ 
TAREAS QUE REFUERCEN EL MATERIAL VISTO EN EL SALÓN DE SESIÓN.
\item[B.]EL PERSONAL DISCENTE RESOLVERÁ PROBLEMAS ES\-PE\-CIA\-LES DE 
APLICACIÓN EN INGENIERÍA DE COMUNICACIONES, REALIZARÁ POR LO MENOS UN 
TRABAJO EXTRA CLASE, PREVIO A CADA UNA DE LAS EVALUACIONES PARCIALES, CON 
EL FIN DE REFORZAR LOS CONOCIMIENTOS TEÓRICOS Y LOS ASPÉCTOS PRÁCTICOS.
\item[C.]PARA COMPLEMENTAR LOS EJES TRANSVERSALES DEL CUR\-SO DE FORMACIÓN 
Y FOMENTAR LA EDUCACIÓN INTEGRAL DEL PERSONAL DISCENTE SE IMPARTIRÁN LOS 
PROGRAMAS DE APOYO EDUCATIVO, QUE TIENEN RELACIÓN CON ESTA ASIGNATURA COMO 
SON DESARROLLO HUMANO, DIFUSIÓN DE LA CULTURA,\ \ \ DE LA COMANDANCIA DEL CUER\-PO, 
TRABAJO PSICOPEDAGÓGICO, TUTORIAL, ENTRE OTROS.
\item[D.]ASIMISMO, PARA EL DESARROLLO ARMÓNICO E INTEGRAL DEL PERSONAL 
DISCENTE CON LA IMPARTICIÓN DE ESTA ASIGNATURA, SE PROMOVERÁN Y POTENCIARÁN 
LOS VALORES DE LEALTAD, HONRADEZ, HONOR, ABNEGACIÓN, ES\-PÍ\-RI\-TU DE CUERPO, 
PATRIOTISMO, VALOR Y DISCIPLINA EN CADA UNA DE LAS ACTIVIDADES EDUCATIVAS.
\end{description}
\item[{\large{\bf IV.}}] {\large{\bf PROCEDIMIENTOS DE EVALUACIÓN.}}\\
SE APLICARAN 3 EVALUACIONES PARCIALES Y UNA E\-VA\-LUA\-CIÓN FINAL ORDINARIA 
UTILIZANDO LA ESCALA DE CALIFICACIÓN DEL 0 AL 10, LA CALIFICACIÓN MÍNIMA 
APROBATORIA ES DE 6,CONSIDERANDO UN VALOR PARA CADA EVALUACIÓN PARCIAL ; 
LA CALIFICACIÓN FINAL SE INTEGRARÁ CON 60 $\%$ DEL PROMEDIO DE LAS 
EVALUACIONES PARCIALES Y EL 40$\%$ DE LA EVALUACIÓN FINAL, DE ACUERDO CON 
LO SIGUIENTE:
\begin{description}
\item[A.]EVALUACIÓN PARCIAL.
\begin{description}
\item[a.]\lineach{EXAMEN ESCRITO}{70 \ \%}
\item[b.]\lineach{PARTICIPACIÓN EN SESIÓN.}{15 $\%$}
\item[c.]\lineach{TRABAJOS EXTRACLASE.}{15 $\%$}
\item[\ \ ]\lineach{TOTAL.}{100 $\%$}
\end{description}
\item[B.]EVALUACIÓN FINAL
\begin{description}
\item[a.]\lineach{EXAMEN ESCRITO.}{100 $\%$}
\item[\ \ ]\lineach{TOTAL.}{100 $\%$}
\end{description}
\item[C.]LA CALIFICACIÓN FINAL DE LA ASIGNATURA SE INTEGRARÁ CON:
\begin{description}
\item[a.]\lineach{PROMEDIO DE LAS EVALUACIONES PARCIALES}{60 $\%$}
\item[b.]\lineach{EXAMEN FINAL ORDINARIO.}{40 $\%$}
\item[\ \ ]\lineach{TOTAL.}{100 $\%$}
\end{description}
\end{description}
DE ACUERDO CON EL ART. 80 PÁRRAFO V, DEL REGLAMENTO DE LA ESCUELA MILITAR 
DE INGENIEROS, QUEDA EXENTO EL PERSONAL DISCENTE; QUE OBTENGA UN PROMEDIO 
MÍNIMO DE 9.0 PUNTOS EN UNA ASIGNATURA, DESPUÉS DE HABER SUSTENTADO LOS 
EXÁMENES PARCIALES DE LA MISMA, A\-SEN\-TÁN\-DO\-SE LA CALIFICACIÓN 
OBTENIDA EN EL PROMEDIO, QUE\-DAN\-DO A ELECCIÓN DEL PERSONAL DISCENTE LA 
OPCIÓN DE PRESENTAR DICHA EVALUACIÓN, PARA INCREMENTAR SU PRO\-ME\-DIO 
GENERAL.

DE ACUERDO CON LA EVALUACIÓN CONTINUA, EN DONDE EL PERSONAL DOCENTE 
CORROBORA EL CUMPLIMIENTO DE LOS OBJETIVOS ESPECÍFICOS; ÉSTA SE REALIZARÁ 
DE ACUERDO CON EL CRITERIO DEL PERSONAL DOCENTE AL FINALIZAR LA SESIÓN, 
A TRAVÉS DE PREGUNTAS ESCRITAS U ORALES O TRABAJOS EXTRASESIÓN, QUE PERMITAN 
OBJETIVAR EL APRENDIZAJE SIGNIFICATIVO, SIENDO REGISTRADA POR EL PERSONAL 
DOCENTE PARA INCORPORAR ESTE REGISTRO A LA E\-VA\-LUA\-CIÓN SUMATORIA DEL 
PERSONAL DISCENTE.


PARA EVALUAR EL ASPECTO AXIOLÓGICO QUE SE RESALTA EN ESTA ASIGNATURA, SE 
EMPLEARÁN INSTRUMENTOS BASADOS EN LA OBSERVACIÓN, TALES COMO ESCALA 
ESTIMATIVA, LISTA DE VERIFICACIÓN O DE COTEJO, ASI COMO EL RE\-GIS\-TRO 
ANECDÓTICO, QUE FORMARÁN PARTE DEONTOLÓGICA\footnote{El ejercicio o la 
puesta en práctica de los valores dentro de cierta profesión, en el caso 
los militares honor, lealtad, patriotismo espíritu de cuerpo etcétera.} 
DEL EJERCICIO PROFESIONAL DE FUTURO INGENIERO MILITAR.

CABE RESALTAR QUE EL ASPECTO AXIOLÓGICO EN ESTE CASO CARECERÁ DE VALOR 
CUANTITATIVO, POR EL CONTRARIO SU VALOR SERÁ CUALITATIVO Y SERVIRÁ COMO 
UN ME\-CA\-NIS\-MO PARA QUE EL PERSONAL DISCENTE RECIBA RETROALIMENTACIÓN 
POR PARTE DEL PERSONAL DOCENTE DE ESTA ASIGNATURA.

LOS EJERCICIOS MILITARES DE APLICACIÓN Y/O PRÁCTICAS DE LA ESPECIALIDAD, 
QUE COADYUVAN AL CUMPLIMIENTO DEL OBJETIVO GENERAL DE ESTA ASIGNATURA, 
SE EVALUARÁN MEDIANTE ESQUEMAS DE INSTRUCCIÓN/EVALUACIÓN Y/O MISIONES Y 
TAREAS DE ADIESTRAMIENTO, LA CALIFICACIÓN FINAL SE OBTENDRÁ DEL PROMEDIO 
DE LOS ESQUEMAS, MISIONES O TAREAS QUE SE DESARROLLEN.

\eject
\item[{\large{\bf V.}}] {\large{\bf BIBLIOGRAFÍA.}}
\begin{description}
\item[A.] BÁSICA.
\begin{enumerate}
\item R. E Ziemer SIGNALS AND SYSTEMS: CONTINUOUS AND DISCRETE 
Macmillan, 3a. Edición, 1993.
\end{enumerate}
\item[B.] COMPLEMENTARIA.
\begin{enumerate}
\item F. J. Taylor PRINCIPLES OF SIGNALS AND SYSTEMS Mcgraw-Hill, 1a. Ed., 
1994. 
\item M.S. Roden ANALOG AND DIGITAL COMMUNICATION SYSTEMS Prentice-Hall, 
3a. Edición, 1991 
\item LOUIS FRENZEL, SISTEMAS ELECTRONICOS DE COMUNICACIONES, ALFAOMEGA. 
2003.
\item ENRIQUE HERRERA, COMUNICACIONES 1. SEÑALES, MODULACION Y 
TRANSMISION, LIMUSA.
\item D.G.E.M. Y RECTORÍA DE LA U.D.E.F.A., ÉTICA Y MORAL MILITAR EN EL 
EJÉRCITO Y FUERZA AÉREA MEXICANOS., ESCUELA DE PENSAMIENTO MILITAR., 2010.
\item MANUAL DE ÉTICA, VALORES Y VIRTUDES MILITARES
\item PROGRAMA DE CAPACITACIÓN Y SENSIBILIZACIÓN PARA EFECTIVOS EN 
PERSPECTIVA DE GÉNERO 2008-2011
\end{enumerate}
\end{description}
\end{description}
\eject
\begin{center}
{\bf OBJETIVO GENERAL.}
\end{center}
\noindent
AL CONCLUIR LA ASIGNATURA EL PERSONAL DISCENTE DISTINGUIRÁ EL COMPORTAMIENTO 
DE SEÑALES Y SISTEMAS LINEALES DISCRETOS TANTO EN EL DOMINIO DEL TIEMPO 
COMO EN EL DOMINIO DE LA FRECUENCIA, EVIDENCIANDO LOS VALORES FUNDAMENTALES 
DEL EJÉRCITO Y FUERZA AÉREA MEXICANO.

\eject
\tableofcontents

\eject
\section{CONCEPTOS BÁSICOS DE SEÑALES Y SISTEMAS.}
%\addcontentsline{toc}{subsection}{A. Clasificación de señales.}
\input tema_I_A.tex

\addcontentsline{toc}{subsection}{B. Clasificación de sistemas.}
\addcontentsline{toc}{subsection}{C. Energía y potencia de señales.}

\eject
\section{ANALISIS DE SEÑALES Y DE SISTEMAS CONTINUOS EN EL DOMINIO DEL TIEMPO.}
\addcontentsline{toc}{subsection}{A. Caracterización de un sistema por 
medio de su ecuación diferencial.}
\addcontentsline{toc}{subsection}{B. La integral de convolución.}
\addcontentsline{toc}{subsection}{C. Respuesta al impulso de un sistema 
lineal continuo.}

\eject
\section{ANÁLISIS DE SEÑALES Y DE SISTEMAS CONTINUOS EN EL DOMINIO DE 
LA FRECUENCIA.}
\addcontentsline{toc}{subsection}{A. Serie trigonométrica de Fourier.}
\addcontentsline{toc}{subsection}{B. Serie exponencial de Fourier.}
\addcontentsline{toc}{subsection}{C. Transformada de Fourier.}
\addcontentsline{toc}{subsection}{D. Propiedades de la transformada de Fourier.}
\addcontentsline{toc}{subsection}{E. Función de transferencia de un sistema 
lineal continuo.}
\addcontentsline{toc}{subsection}{{\bf PRIMER EXAMEN PARCIAL.}}
\addcontentsline{toc}{subsection}{{\bf REVISIÓN DE LA EVALUACIÓN}}

\eject
\section{MODULACION DE SEÑALES.}
\addcontentsline{toc}{subsection}{A. Modulación de amplitud.}
\addcontentsline{toc}{subsection}{B. Variantes de modulación de amplitud.}
\addcontentsline{toc}{subsection}{C. Multicanalización por división de frecuencia.}
\addcontentsline{toc}{subsection}{D. Recepción de señales moduladas en amplitud.}
\addcontentsline{toc}{subsection}{E. Modulación de frecuencia.}
\addcontentsline{toc}{subsection}{F. Recepción de señales moduladas en frecuencia.}
\addcontentsline{toc}{subsection}{{\bf SEGUNDO EXAMEN PARCIAL.}}
\addcontentsline{toc}{subsection}{{\bf REVISIÓN DE LA EVALUACIÓN.}}

\eject
\section{INTRODUCCION AL DISEÑO DE FILTROS ANALÓGICOS.}
\addcontentsline{toc}{subsection}{A. Filtros ideales.}
\addcontentsline{toc}{subsection}{B. Ancho de banda y tiempo de elevación.}
\addcontentsline{toc}{subsection}{C. Aproximación de filtros ideales con 
filtros realizables.}
\addcontentsline{toc}{subsection}{D. Filtros Butterworth.}
\addcontentsline{toc}{subsection}{E. Filtros Chebyshev.}
\addcontentsline{toc}{subsection}{{\bf TERCER EXAMEN PARCIAL.}}
\addcontentsline{toc}{subsection}{{\bf REVISIÓN DE LA EVALUACIÓN}}
\addcontentsline{toc}{subsection}{{\bf EXAMEN FINAL ORDINARIO.}}
\eject
\addcontentsline{toc}{subsection}{Referencias}
\input bibliografia.tex

\end{document}
