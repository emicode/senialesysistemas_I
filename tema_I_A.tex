
La ingeniería eléctrica y computacional está encargada de solucionar dos tipos de
problemas: (1) la producción o transmisión de energía eléctrica y (2) la transmisión o 
procesamiento de información. Los sistemas de comunicación están diseñados para 
transmitir información.

Es importante entender que los sistemas de comunicación y los sistemas de energía eléctrica
tienen un conjunto de restricciones muy marcadas. Las formas de onda en los sistemas de energía
eléctrica son generalmente conocidas, además de que el interés radica en el diseño del sistema para
que presente una mínima pérdida de energía.

Las formas de onda en los sistemas de comunicación presentes en el receptor (usuario) son
desconocidas hasta que se reciben, de otra manera no se transmitiría información alguna ni habría
necesidad del sistema de comunicación.

Los sistemas de comunicación están diseñados para transmitir a los receptores información
que contiene formas de onda. Existen muchas posibilidades de seleccionar formas de onda para
presentar la información.

En los sistemas de comunicación, la forma de onda recibida es usualmente dividida en parte 
deseada que contiene la información o señal, y en la parte residual o indeseada, llamada ruido.
Las formas de onda se representarán mediante expresiones matemáticas
directas, o con series ortogonales, como las de Fourier. Las propiedades de estas formas de 
onda tales como su valor de DC, su valor cuadrático medio (RMS), potencia normalizada, espectro 
de magnitud, espectro de fase, densidad espectral de potencia y ancho de banda son también 
establecidas. Además se estudiarán los efectos de la filtración lineal.

La forma de onda de interés puede ser el voltaje como una función del tiempo, $v(t)$, 
o la corriente como una función del tiempo, $i(t)$. Las mismas técnicas matemáticas 
pueden utilizarse cuando se trabaja con cualquiera de estos tipos de forma de onda. Por 
lo tanto, como enfoque general, las formas de onda se representarán por medio de $w(t)$ 
cuando el análisis se aplique a cualesquier caso.
\eject
\noindent{\bf Formas de onda físicamente realizables}\\
Las formas de onda prácticas que son físicamente realizables (es decir, medibles en un laboratorio)
satisfacen varias condiciones:
\begin{enumerate}
\item La forma de onda tiene valores significativos diferentes a cero sobre un intervalo compuesto
de tiempo f\/inito.
\item El espectro de la forma de onda tiene valores significativos sobre un intervalo compuesto de
frecuencia f\/inito.
\item La forma de onda es una función continua en el tiempo.
\item La forma de onda tiene un valor pico f\/inito.
\item La forma de onda sólo tiene valores reales. Esto es, en cualquier momento no puede tener un
valor complejo de $a + jb$, donde b es diferente de cero.
\end{enumerate}
La primera condición es necesaria debido a que los sistemas (y sus formas de onda) parecen
existir por una cantidad finita de tiempo; las señales físicas también producen una cantidad finita de
energía. La segunda se requiere debido a que cualquier medio de transmisión, ya sean cables, cables
coaxiales, guías de onda o cable de fibra óptica, tienen un ancho de banda restringido. La tercera con-
dición es una consecuencia de la segunda y se hará más clara con la ayuda del análisis espectral. 
La cuarta es menester porque los dispositivos físicos se destruyen si
está presente un valor infinito de voltaje o de corriente dentro del dispositivo. La quinta condición
resulta del hecho de que sólo las formas de onda reales pueden observarse en el mundo real a pesar
de que las propiedades de una forma de onda, como los espectros, pueden ser complejos

Los modelos matemáticos que no cumplen alguna o todas las condiciones mencionadas se utilizan por 
una razón principal: simplificar el análisis ma\-te\-má\-ti\-co. Sin embargo, si se es cuidadoso
con el modelo matemático, puede obtenerse el resultado correcto cuando se interpreta la respuesta
adecuadamente.
Por ejemplo, considere que la forma de onda digital mostrada en la figura 2-1 (pag. 35 de 
\cite{Couch}) presenta discontinuidades durante los tiempos de conmutación. Esta situación no 
cumple la tercera condición, aquella sobre la necesidad de que la forma de onda física sea continua. 
La forma de onda física tiene una duración finita (decae a cero antes de $t = \pm\infty$), pero 
la duración de la forma de onda matemática se extiende hasta el infinito.

En otras palabras, el modelo matemático asume que la forma de onda física ha existido en su
condición de estado estable durante todo el tiempo. El análisis espectral del modelo aproximará los
resultados correctos, excepto para los componentes con frecuencias extremadamente altas. La potencia 
promedio que se calcula del modelo resultará en el valor correcto para la potencia promedio de
la señal física que se medirá durante un intervalo de tiempo adecuado. La energía total de la señal
del modelo matemático será infinita porque se extiende a un tiempo infinito, mientras que la de la
señal física será finita. Por consiguiente, el modelo no generará el valor correcto para la energía 
total de la señal física sin utilizar alguna información adicional. Sin embargo, el modelo puede 
utilizarse para calcular la energía de la señal física durante un intervalo de tiempo finito. 
Se dice que el modelo matemático es una señal de potencia debido a que tiene la propiedad de 
potencia finita (y energía infinita), mientras que la forma de onda física es una señal de energía 
debido a que tiene energía finita. (Las definiciones matemáticas para las señales de potencia y 
energía se presentarán 
más adelante.) Todas las señales físicas son señales de energía, aunque a menudo se utilizan 
modelos matemáticos de señales de potencia para simplificar el análisis. 
\ \\\ \\
%\subsection{Clasificación de señales}
\addcontentsline{toc}{subsection}{A. Clasif\/icación de señales.}
\noindent{\bf {\large A. Clasif\/icación de señales.}}\par
En resumen, las formas de onda se clasifican como señales o ruido, digitales o analógicas, 
determinísticas o no determinísticas, físicamente realizables o físicamente no realizables y 
pertenecientes a dos tipos: de potencia o de energía. La siguiente sección presentará 
clasificaciones adicionales, como periódicas y no periódicas.\par
\noindent Nota de clase 2015.03.05\\
\noindent Chavez Hernández Sergio Yair\\
\noindent Contreras Díaz Iván\\





